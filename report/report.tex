\documentclass[11pt]{article}

\usepackage[utf8]{inputenc}
\usepackage[english]{babel}
\usepackage{graphicx}
\usepackage[margin=1in]{geometry}
\usepackage[parfill]{parskip}

\title{LPHY1300 Personal project report :\\ Simulation of a particle detector}
\author{\textsc{Schils} Arnaud\\ \textsc{Lardinois} Simon}
\date{2015-2016}

\begin{document}

\maketitle
\newpage
\renewcommand{\contentsname}{Table of contents}
\tableofcontents




\newpage
\section*{Introduction}

Nowadays, the huge amount of computing power offered by modern computer systems allows
to solve complex scientific problems numerically. The art of solving physics problems using
computers is referred as \textit{computational physics}. This modern field
is multidisciplinary: it brings together applied mathematics,
computer science and physics. If offers a third way to do physics
that supplements theory and experiment.

In this thesis, these numerical techniques are applied to the field of particle
detectors physics. In order to design the best detectors,
physicists have to know the measured signal resulting from the passage of a particle
in the detector. Unfortunately, due to the complex geometries of these detectors
and the various physical phenomena to handle, this signal is not easy
to compute analytically.

It is why, in the context of this thesis, a software  to compute the current measured
by a particle detector due to the passage of a particle has been developed.
This software has then been used to study silicon and gas particle detectors.

	% In the context of our lesson LPHY1300 personal project, we
	% had to simulate a particle detector using the finite elements method. We
	% thus develop our program "pdetect" which simulates the electric potential
	% and electric field inside a pixel detector. It also computes the electric
	% current induce by an incident particle with a given trajectory.

	\subsection*{Overview of the contributions}

	The primary contribution of this thesis is the development of a C++ numerical software performing the
	following tasks.

	Firstly, it computes the potential and the electric field
	at each point of a particle detector solving the Laplace equation for non-trivial
	two-dimensions geometries. Three types of 2D detector geometries are supported.
	The Laplace equation is solved using the \textit{finite element method} with an adaptive
	grid refinement strategy. The software supports multithreading and therefore
	uses all available CPUs to quickly solve this equation.

	Secondly, the software computes the current resulting from the passage of a particle
	in the detector using the \textit{Shockley–Ramo theorem} and the solution of the
	computed electric field. Every possible particle trajectories in the detector
	are suppoted. Effects such as the \textit{mobility} difference between holes and
	electrons, the \textit{saturation} phenomena and the \textit{Townsend Avalanche}
	are handled.

	This software has been developed following the object-oriented
	programming paradigm and with the \textit{model-view-controller}
	software engineering pattern in mind. Thanks to the quality of the software
	architecture further extensions such as 3D geometries, additional physical effects or
	detector types and the support of distributed computations on clusters are possible
	without the burden of reimplementing an entire new software.

	The secondary contribution of this thesis is the use of this software to study
	the gas and silicon detectors. Finally, the results provided by the
	software have been compared with the results of \textit{Weightfield}~\cite{Cenna2015}, a
	software performing similar computations.

	%To perform this task, the software places charges
	%resulting from the ionization of the molecules inside the detector. Then, these
	%charges drift following the applied potential at the anode and cathode of the detector

	\subsection*{Organization of the thesis}

	TODO

	% In this paper we will first introduce briefly what the finite elements
	% method is, and then a library we used in our program, "deal.II".
	% After what we will explain properly our program, the way it works, and
	% also comparing our results with the results of "weightfield", another
	% simulator of a particle detector, already use in many experiences.
	% To finish this paper we will talk about two specific cases of detector,
	% a silicon detector, and a helium detector.


\section{Background and related work}

This section is composed of three parts. Firstly,
 the finite element method is briefly explained. Then \textit{deal.ii},
 a C++ library implementing the finite element method is introduced. Finally,
 \textit{Weightfield}, another software performing particle detector simulation,
 is presented.

	\subsection{Finite element method}

		The finite element method is a numerical technique for
		finding approximate solutions to boundary value problems for partial
		differential equations. The finite element method subdivides a large
		problem into smaller, simpler, parts, called finite elements. The
		simple equations that model these finite elements are then assembled
		into a larger system of equations that models the entire problem. It
		then uses variational methods from the calculus of variations to
		approximate a solution by minimizing an associated error function.

	\subsection{Deal.II library}

	\subsection{Weightfield}

\section{Introduction to the physics of particle detectors}

\section{Software architecture}

\section{Results for silicon and gas detectors}

\section{Future work}

\section{Conclusion}

\newpage

\bibliographystyle{plain}
\bibliography{report.bib}

\end{document}
