\documentclass[11pt]{article}

\usepackage[utf8]{inputenc}
\usepackage[english]{babel}
\usepackage{graphicx}
\usepackage[margin=1in]{geometry}
\usepackage[parfill]{parskip}

\title{LPHY1300 Personal project report :\\ Simulation of a particle detector}
\author{\textsc{Schils} Arnaud\\ \textsc{Lardinois} Simon}
\date{2015-2016}

\begin{document}

\maketitle
\newpage
\renewcommand{\contentsname}{Table of contents}
\tableofcontents




\newpage
\section*{Introduction}

	In the context of our lesson LPHY1300 personal project, we
	had to simulate a particle detector using the finite elements method. We
	thus develop our program "pdetect" which simulates the electric potential
	and electric field inside a pixel detector. It also computes the electric
	current induce by an incident particle with a given trajectory.

	In this paper we will first introduce briefly what the finite elements
	method is, and then a library we used in our program, "deal.II".
	After what we will explain properly our program, the way it works, and
	also comparing our results with the results of "weightfield", another
	simulator of a particle detector, already use in many experiences.
	To finish this paper we will talk about two specific cases of detector,
	a silicon detector, and a helium detector.

	\subsection*{Goals}

	\subsection*{Overview of the Contributions}

	\subsection*{Organization of the Thesis}

\section{Background and related work}

	\subsection{Finite elements method}

		The finite element method is a numerical technique for
		finding approximate solutions to boundary value problems for partial
		differential equations. The finite elements method subdivides a large
		problem into smaller, simpler, parts, called finite elements. The
		simple equations that model these finite elements are then assembled
		into a larger system of equations that models the entire problem. It
		then uses variational methods from the calculus of variations to
		approximate a solution by minimizing an associated error function.

	\subsection{Deal.II library}

	\subsection{Weightfield}

\section{Introduction to particle detector physics}

\section{Software architecture}

\section{Results for silicon and gas detectors}

\section{Future work}

\section{Conclusion}

\end{document}
